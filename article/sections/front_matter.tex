\begin{center}
    \makeatletter

        {\Large Trabalho de Linguagens Formais e Autômatos \par}

        {\LARGE \@title \par}

        \vspace{0.5cm}
        {
            \large \@author \\
            \small \texttt{<arielnogueirak@gmail.com>}
            \par
        }

        \vspace{0.5cm}
        \textit{
           Computer Engineering Course, \\
           Instituto Politécnico (IPRJ) --- Rio de Janeiro State University, \\
           Rua Bonfim 25, Nova Friburgo, RJ 28625-570, Brazil
        }

        \vspace{0.5cm}
        \@date
    \makeatother

    \vspace{0.5cm}
    \hrule
\end{center}

\section*{Abstract}

In this assignment we build and analyze the behavior and output of a Finite
State Machine (FSM) and a Turing Machine (TM) for a given set of inputs.

\medskip
\noindent
\textit{Keywords:} finite state machine, turing machine, finite automata,
formal languages, chomsky hierarchy

\section*{Resumo}

Neste trabalho nós construímos e analisamos o comportamento e a saída de uma
Máquina de Estado Finito (MEF) e uma Máquina de Turing (MT) para um dado
conjunto de entradas.

\medskip
\noindent
\textit{Palavras-chave:} máquina de estado finito, máquina de turing, autômatos
finitos, linguagens formais, hierarquia de chomsky